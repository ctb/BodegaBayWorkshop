

\documentclass{beamer}
 
\usepackage[utf8]{inputenc}
 
 
%Information to be included in the title page:
\title{Processing ddRAD for population history inference}
\author{April Wright}
\institute{ISU, KU}
\date{Jan 6, 2016!}
 
 
 
\begin{document}
 
\frame{\titlepage}
 
\begin{frame}
\frametitle{ddRAD data}
\begin{itemize}
\item Reduced-representation genomic method
\end{itemize}
\end{frame}

\begin{frame}
\frametitle{ddRAD data}
\begin{itemize}
\item Reduced-representation genomic method
\item Cheap
\end{itemize}
\end{frame}

\begin{frame}
\frametitle{ddRAD data}
\begin{itemize}
\item Reduced-representation genomic method
\item Cheap
\item Lots of data returned
\end{itemize}
\end{frame}

\begin{frame}
\frametitle{ddRAD data}
\begin{itemize}
\item Reduced-representation genomic method
\item Cheap
\item Lots of data returned
\end{itemize}
\end{frame}
 
\begin{frame}
\frametitle{ddRAD data}
\begin{itemize}
\item Reduced-representation genomic method
\item Cheap
\item Lots of data returned
\item Stable software pipelines for using these data
\end{itemize}
\end{frame}

\begin{frame}
\frametitle{A Quick Note}
Slides that contain ddRAD specific info will be noted. Some steps can be used with multiple data sources.
\end{frame}

\begin{frame}
\frametitle{Our Study}
Eurycea
\end{frame}

\begin{frame}
\frametitle{Our Study}
Eurycea
Development
\end{frame}

\begin{frame}
\frametitle{Our Study}
Eurycea
Development
Prior work
\end{frame}

\begin{frame}
\frametitle{Our Study}
Eurycea
Development
Our work
\end{frame}

\begin{frame}
\frametitle{Phylogenetics}
\begin{itemize}
\item Maximum likelihood
\item Cheap
\end{itemize}
\end{frame}

\begin{frame}
\frametitle{Phylogenetics}
\begin{itemize}
\item Maximum likelihood
\item Statistically consistent
\end{itemize}
\end{frame}

\begin{frame}
\frametitle{Phylogenetics}
\begin{itemize}
\item Maximum likelihood
\item Statistically consistent
\item Superimposed changes
\end{itemize}
\end{frame}

\begin{frame}
\frametitle{Phylogenetics}
\begin{itemize}
\item Maximum likelihood
\item Statistically consistent
\item Superimposed changes
\item Model-based 
\end{itemize}
\end{frame}

\begin{frame}
\frametitle{Phylogenetics}
\item Problems
\end{frame}

\begin{frame}
\frametitle{Phylogenetics}
\begin{itemize}
\item Problems
\item Missing data
\end{itemize}
\end{frame}

\begin{frame}
\frametitle{The Demultiplex}
One of the things that makes RADseq, and especially ddRADseq so cheap is the pooling of samples
\end{frame}

\begin{frame}
\frametitle{The Demultiplex}
One of the things that makes RADseq, and especially ddRADseq so cheap is the pooling of samples
The way we recover individual samples is via demultiplexing
\end{frame}

\begin{frame}
\frametitle{The Demultiplex}
ddRAD data include an inline barcode and an Illumina index. This allows multiple pools to be combined, and individuals to be separated back out computationally. 
This allows for the cost-saving properties of batching, without the cost-increasing properties of synthesizing oligonucleotides.
\end{frame}

\begin{frame}
\frametitle{The Demultiplex}
We'll be using STACKS for this step
\end{frame}

\begin{frame}
\frametitle{The Demultiplex}
We'll be using STACKS for this step
\end{frame}

\begin{frame}
\frametitle{The Demultiplex}
We'll be using STACKS for this step
\end{frame}

\begin{frame}
\frametitle{The Demultiplex}
We'll be using STACKS for this step
\end{frame}

\begin{frame}
\frametitle{The Demultiplex}
Break: exercise
\end{frame}

\begin{frame}
\frametitle{Building Individual Stacks}
What is this?
\end{frame}

\begin{frame}
\frametitle{Building Individual Stacks}
Break: exercise
\end{frame}

\begin{frame}
\frametitle{Building Individual Stacks}
Break: exercise
\end{frame}

\begin{frame}
\frametitle{Building a Catalog}
Break: exercise
\end{frame}

\begin{frame}
\frametitle{Building a Catalog}
Break: exercise
\end{frame}

\begin{frame}
\frametitle{Building a Catalog}
Break: exercise
\end{frame}

\begin{frame}
\frametitle{Mapping Individuals to the Catalog}
Why?
\end{frame}

\begin{frame}
\frametitle{Mapping Individuals to the Catalog}
Why?
\end{frame}

\begin{frame}
\frametitle{Mapping Individuals to the Catalog}
Break
\end{frame}

\begin{frame}
\frametitle{Output}
How?
\end{frame}

\begin{frame}
\frametitle{Output}
How?
\end{frame}

\begin{frame}
\frametitle{What do our missing data look like?}
UNIX scripts to see how much per line
\end{frame}

\begin{frame}
\frametitle{What do our missing data look like?}
PAUP and pairs
\end{frame}

\begin{frame}
\frametitle{What do our missing data look like?}
Let's build a tree
\end{frame}


\end{document}


